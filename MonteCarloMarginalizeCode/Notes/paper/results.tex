\label{sec:Results}

% POINT: What events were used
To provide a stringent, production-environment test of our parameter estimation code, we applied it to \nEventsMDC{}
events from the 2015 BNS mock data challenge, described below.  
%
% POINT: What code was used
Unless otherwise stated, we adopted nonspinning TaylorT4 templates at 3.5 PN order, including only the $(2,\pm 2)$ and
$(2,0)$ modes;  used roughly $200$ mass points for each event, distributed on spokes in $\mc,\eta$ radiating away from the reported
masses  provided by the searches, out to an estimated match of 0.9; 
 evaluated the Monte Carlo integral $10$ times for each mass point; and used a sampling prior equal to the prior for all
 parameters except distance and mass, which were set adaptively and \textbf{via a skymap}.    As used here, the skymap had
 \textbf{$12\times 64^2$ pixels, roughly one per square degree.}  Our distance prior was
 uniform in volume out to $d=300\unit{Mpc}$.  
% POINT: Data handling choices
For each event, the code uses the \gstlal{} input noise power spectrum in each instrument; starts the waveform at $f_{\rm
  low}=40\unit{Hz}$\footnote{ \textbf{Why is this not a command-line option of ILE, tied to data selection in CME?}};
and uses an inverse power spectrum filter targeting the frequency range $[f_{\rm min},f_{\rm max}]=[f_{\rm min},
  2000\unit{Hz}]$, constructed from the measured power spectrum as described in Appendix \ref{ap:DiscreteData}.  

\subsection{2015 BNS MDC}


For reference, events in the 2015 MDC data were uniformly distributed on the sky, in orientation, and in volume out to
\textbf{219} \unit{Mpc}.  The BNS injections had random masses in $1.2 M_\odot-1.6 M_\odot$ and randomly oriented spins with
\textbf{ZZZ}.   The \gstlal{} pipeline was used to select events for further study.  This selection process introduces a
small selection (Malmquist) bias, described in Figure \ref{fig:SearchSelection}, 
which slightly disfavors edge-on binaries relative to our prior.    
%

Our parameter estimation strategy does not account for selection biases; for a sufficiently large ensemble of events,
small deviations between the statistical properties of our posteriors and the ensemble are expected. 


\begin{figure}
\includegraphics[width=\columnwidth]{../Figures/fig-mma-manual-2015MDC-SelectedEvents-DistanceCumulative}
\includegraphics[width=\columnwidth]{../Figures/fig-mma-manual-2015MDC-SelectedEvents-CosIotaCumulative}
\caption{\label{fig:SearchSelection}\textbf{Selection biases}: For the 450 (\textbf{currently 120}) events used in our followup study, the cumulative distribution of $d^3$
  and $\cos \iota$.  In the absence of selection biases and in the limit of many samples, these two parameters should be
  uniformly distributed; small deviations away from uniformity reflect selection biases and sampling error.
}
\end{figure}


\subsection{Scaling}

* scaling versus $f_{\rm low}$



\ForInternalReference{
* scaling versus number of harmonics used.  (Aside on truncating harmonics with trivial content...not used presently)
}


\ForInternalReference{
\begin{table}
\begin{tabular}{lll}
$f_{\rm low}$ & $t_{\rm wave}$ & $T_{\rm wall}$ \\\hline
10 & & \\
25 & & \\
30 & & \\
\end{tabular}
\caption{\textbf{Runtime versus starting frequency}: Waveform duration $t_{\rm wave}$ and wallclock time $T_{\rm wall}$  needed to evaluate $L_{\rm red}=\int L p d\theta$
  for one set of intrinsic parameters $\lambda$ versus starting frequency $f_{|rm low}$, for a $m_1,m_2=\textbf{XXX}$
  nonspinning black hole binary.  Waveforms were generated using the  \texttt{TaylorT1} and \texttt{EOBNRv2} time-domain
  codes, respectively.  The
  computational cost does not depend significantly on waveform duration for starting frequencies of interest. 
}
\end{table}
}

\begin{figure}
\includegraphics[width=\columnwidth]{../Figures/fig-manual-RuntimeScalingVsFmin.png}
\caption{\textbf{Scaling versus frequency}: Points show the runtime of our parameter estimation strategy as a function
  of the minimum frequency $f_{\rm min}$.  For comparison, the solid curve shows the scaling $\propto f_{\rm
    min}^{-8/3}$ expected if our runtime was proportional to the waveform duration (e.g., runtime proportional to the
  number of time samples). 
% INTERNAL REF: coinc_id_17494 was used here.
 Waveforms were generated using the standard \texttt{TaylorT4} time-domain code, with $m_1=1.55 M_\odot$ and $m_2=1.23 M_\odot$. 
}
\end{figure}

\subsection{Detailed investigation of one event}
%% WHAT FIGURES CURRENTLY ARE : 
%     - https://ldas-jobs.phys.uwm.edu/~evano/skymap_reruns/coinc_id_833/ 
%     : v2 = bayestar as prior and sampler, adapt in distance
%    
%% m1 = 1.21992194653 (Msun)
%% m2 = 1.20069205761 (Msun)
%% s1x = 0.00190996297169
%% s1y = 0.00721042277291
%% s1z = -0.00683548022062
%% s2x = -0.00129437400028
%% s2y = -7.87806784501e-05
%% s2z = 0.00192063604482
%% lambda1 = 0.0
%% lambda2 = 0.0
%% inclination = 2.62300610542
%% distance = 112.5338974 (Mpc)
%% reference orbital phase = 4.29952907562
%% time of coalescence = 966822123.762
%% detector is: H1
%% Sky position relative to geocenter is:
%% declination = 0.678935229778 (radians)
%% right ascension = 4.10562181473 (radians)
%% polarization angle = 6.19411420822

Despite providing complete results in less than one hour, our strategy provides extremely well-sampled distributions and
evidence, with small statistical error.   To illustrate its performance, we have selected a single event, whose
parameters are provided in Table \ref{tab:FiducialEvent:Parameters}.   
%



% POINT: Consistency 
The repeated independent evaluations naturally produced by our algorithm provide a simple self-consistency check,
allowing us to quantify convergence empirically.  
Specifically, at each of the 111 mass points automatically targeted for investigation by our algorithm for this event, we
independently evaluated the integral $L_{\rm red}$ [Figure \ref{fig:FiducialEvent:LikelihoodVersusMchirpEta}] and construct one- and two-dimensional posterior distributions, from
10 independent evaluations [e.g., Figure \ref{fig:FiducialEvent:Triplot:TriggerMasses}].    
%
As illustrated by example in Figure \ref{fig:FiducialEvent:Triplot:TriggerMasses}, these fixed-mass posterior
distributions are smooth and overlap the true parameters \editremark{Need to add injection cross}.  
%
Moreover, as illustrated by Figure \ref{fig:FiducialEvent:Cumulatives:Comparison:TriggerMasses}, each of the 10
independent evaluations make posterior predictions that are consistent with one another.   
%
Finally, as illustrated by Figure \ref{fig:FiducialEvent:Integral:ErrorEstimate}, for each mass point the 10 values of $L_{\rm red}$
are consistent with one another to roughly $1\%$.  
%
Keeping in mind our final reported results combine both all mass points and all 10 evaluations at each mass point, we
anticipate relatively little uncertainty due to sampling error in our current configuration.  


\begin{table}
\begin{tabular}{l|ll}
Parameter & True & Search \\ \hline
$m_1 (M_\odot)$ &  1.22 & 1.26 \\
$m_2 (M_\odot)$ &  1.20 & 1.16 \\
$|\chi_1| $ & 0.01  & 0 \\
$|\chi_2| $ & 0.002 & 0 \\
$d (\unit{Mpc}) $ & 112.5 & - \\
$\iota $ & 2.62 & - \\
($\alpha,\delta$) & (4.106,0.6789) &\\ 
$\rho_{\rm search}$ & 10.85 \\
\end{tabular}
\caption{\label{tab:FiducialEvent:Parameters}\textbf{Fiducial event: True and trigger parameters}: The physical parameters of our injected event, compared
  with the parameters provided by the search and used to target our parameter estimation followup.
}
\end{table}




\begin{figure*}
\includegraphics[width=\textwidth]{../Figures/v2runs_coinc_id_833_ILE_triplot_MASS_SET_0}
\caption{\label{fig:FiducialEvent:Triplot:TriggerMasses}\textbf{Posterior distribution in intrinsic parameters, assuming known masses}: For our fiducial event, our predicted
  distribution of extrinsic parameters $d,RA=\alpha,DEC=\delta,\iota,t,\phi,\psi$, for clarity evaluated assuming at the
  mass parameters identified by the search.  Extremely similar distributions are recovered at each mass point. 
\ForInternalReference{  \emph{Suggest: show d-cos iota, skymap, and phi-psi only, not full triplot}}
}
\end{figure*}


\begin{figure}
%\includegraphics[width=\columnwidth]{../Figures/v2runs_coinc_id_833_mchirp_eta_logevidence}
\includegraphics[width=\columnwidth]{../Figures/fig-mma-manual-coinc833-LReducedVersusMcEta}
\caption{\label{fig:FiducialEvent:LikelihoodVersusMchirpEta}\textbf{Marginalized likelihood and posterior distribution
    versus component masses}: \emph{Top panel}: 
For our fiducial event,
  contours of the log of integrated likelihood $\ln \LikeRed$ versus component masses, represented in $\mc,\eta$
  coordinates.  Points indicate the mass grid adopted; thin contours show isocontours of the (interpolated)
  $\ln \LikeRed=35,36,37,38,39$.   For comparison, the thick black curve corresponds to the 90\% confidence interval
  predicted by combining the network SNR reported by the search ($\rho_{\rm search}$ in Table \ref{tab:FiducialEvent:Parameters}) and the (effective)
  Fisher matrix used when placing test points.  
\emph{Bottom panel}: The 90\% confidence interval derived from $L_{\rm red} p(\mc,\eta)$.  
 \editremark{Once OK with Ben Farr, add curves from MCMC TaylorF2}
}
\end{figure}



\begin{figure}
\includegraphics[width=\columnwidth]{../Figures/v2runs_coinc_id_833_cumulative-multiplot-distance-MASS_SET_0}
\caption{\label{fig:FiducialEvent:Cumulatives:Comparison:TriggerMasses}\textbf{Sampling error analysis I: One-dimensional cumulative distribution at fixed mass}:  For each of the 10 independent
  instances used at  the mass  parameters   identified by the search, a plot of the one-dimensional cumulative
  distribution in distance.  These distributions agree to within a few \textbf{(quantify)} percent, qualitatively
  consistent with a naive estimate based on $1/\sqrt{n_{\rm eff}} \simeq 4\%$.   Combining all 10
  independent runs, we expect the final distance posterior has even smaller statistical sampling error
  (\textbf{quantify} $\simeq X/\sqrt{10}$).  Our final posterior
  distributions, having support from several mass points, should have smaller statistical error still.
 \editremark{Once OK with Ben Farr, add curves from MCMC TaylorF2}
}
\end{figure}

\begin{figure}
\includegraphics[width=\columnwidth]{../Figures/fig-mma-manual-v2_coinc_833-IntegralVarianceVersusNeff}
\caption{\label{fig:FiducialEvent:Integral:ErrorEstimate}\textbf{Sampling error analysis II: Integral error}: For each of the 111 mass points evaluated for the fiducial
  event, a scatterplot of the mean number of effective samples $n_{\rm eff}$ versus the standard deviation in $\ln
  L_{\rm red}$, where at each mass point the mean and standard deviation are calculated over the 10 independent
  evaluations performed.  Considering our final prediction for $L_{\rm red}$ combines all 10 events, this figure
  suggests $L_{\rm red}$ is known to better than $1\%$ for each mass point.  
% Additionally, this figure illustrates just how many effective samples are available for *each* mass point: thousands.
}
\end{figure}

\subsection{Ensemble of events}

If our estimates for the one-dimensional cumulative distributions $P(<x)$ are unbiased and if $x_*$ is a random variable
consistent with the prior, then $P(x_*)$ should be a uniformly-distributed random variable.   To test this hypothesis,
we use the one-dimensional posteriors provided by the MDC.
%% we perform repeated simulations, where each injected event was drawn from our prior:
%% \begin{itemize}
%% \item 2015 BNS MDC:   We selected \nEventsMDC{} events from the 2015 BNS MDC, identified by the \gstlal{} pipeline.  While all events
%%   have two-dimensional skymaps produced by \BS{}, the analysis presented below did \textbf{not} use two-dimensional skymaps.

%% While the NS-NS binaries in the 2015 MDC had generic spins, our parameter estimation model assumed zero spin.
%% \end{itemize}

% POINT: pp plots for an ensemble of events
For each parameter $x$, each colored curve in Figure  \ref{fig:pp:2015Ensemble} is  the fraction of events with
estimated cumulative probability $P(<x_*)$ at the injected parameter value $x_*$.  
Specifically, if $P(x_{*q})$ are the sorted cumulative probabilities for the $q=1\ldots n$ events with
$P(x_{*1})<P(x_{*2})$, then the points on the plot are $\{P(x_{*,q}),q/n\}$.  
%

\begin{figure}
%\includegraphics[width=\columnwidth]{../Figures/2015_BNS_MDC_pat_and_chris_pp_plot}   % Original runs
\includegraphics[width=\columnwidth]{../Figures/v2_2015_BNS_MDC_skysampling_pp_plot}  % Bayestar as prior and sampling prior
\caption{\label{fig:pp:2015Ensemble}\textbf{PP plot for ensemble of XXX NS-NS events}: \emph{Top panel}: For \textbf{X} randomly-selected NS-NS binaries, a plot of
  the cumulative distribution of $P_\theta(\theta_k)$ for each extrinsic variable $\theta=d,RA,DEC,\iota,\psi,\phi_{\rm
    orb}$.
\emph{Bottom panel}: The sky area associated with higher-probability pixels than the true sky position of the source. \textbf{CREATE}
}
\end{figure}

