
We begin with the division of the parameter space of compact binary coalescence waveforms. We first define the difference between ``intrinsic'' and ``extrinsic'' parameters. We define $\vec{\lambda}$ as the set of intrinsic parameter as those corresponding to the physical confiugration of the binary: $\vec{\lambda}=\{\mc,\eta,\Lambda_1,\Lambda_2,\vec{S}_1,\vec{S}_2,\vec{L}\}$, where $\mc$ and $\eta$ are the chirp mass () and symmetric mass ratio (), $\Lambda_1$ and $\Lambda_2$ are the tidal numbers of each component mass\footnote{Black holes have no tidal number and thus $\Lambda=0$}, $\vec{S}_1$ and $\vec{S}_2$ correspond to the spin vectors of the component compact objects, and finally $\vec{L}$ is the overall angular momentum vec of the system. We emphasize in this work rapid determination of source masses --- and possibly compact object type --- thus focusing attention exclusively on $\vec{\lambda}={\mc,\eta}$. While the impact of spin and tidal parameters on the waveform are important, they are also potentially complicate the computation of the likelihood in the scheme described later. We thus leave the remaining intrinsic parameters to be built upon in future work.

The extrinsic parameters can be interpreted as the effect of the waveform's arrival and response on a given gravitational-wave detector. For a set of gravitational-wave interferometers, a seven-dimensional space of extrinsic parameters is defined as $\vec{\theta}={t_c,\alpha,\delta,\iota,D,\psi,\phi}$, where 
